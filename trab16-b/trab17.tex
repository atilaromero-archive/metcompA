
%%% cabecalho %%%
\documentclass[brazilian,12pt,a4paper,final]{article}
\usepackage[brazil]{babel}% *babel* contem as regras de ifenização
%\usepackage{t1enc}% *t1enc* permite o reconhecimento dos acentos do teclado
\usepackage[utf8]{inputenc}% permite reconhecimento automático de acentuação.
\usepackage{graphicx} % para incluir figuras em formato eps 
%ou
%\usepackage[pdftex]{graphicx}% para produzir PDF diretamente
\usepackage{color} % fontes soloridas
%%% fim do cabecalho %%%

\pagestyle{empty}
\title{Métodos Computacionais da Física A \\ Trabalho 15}
\author{Aluno: Átila Leites Romero \\ Matrícula: 144679 \\ IF-UFRGS}

\begin{document}
\maketitle

$f(x) = rx(1-x)-x$

$f(x) = -rx^2 + rx -x $

$f(x) = x(-rx + r -1)$

logo, ou

$x = 0$

ou

$-rx + r -1 = 0$

$-rx + r = 1$

$r(-x + 1) = 1$

$-x = \frac{1}{r}-1$

$x = 1 - \frac{1}{r}$

Mas $F(x)= f(x)+x$, ou seja, $F(x) = -rx^2 + rx$. A condição para que o método
de interação simples funcione é que o módulo da derivada de $F(x)$ quando $x = 0$
precisa ser $<= 1$.

$\frac{\delta F(x)}{\delta x} = -2rx + r$

$|\frac{\delta F(x)}{\delta x}| <= 1$

$|-2rx + r| <= 1$

E como as raízes são $x = 0$ e $x = 1 - \frac{1}{r}$, as condições se tornam:

$r <= 1$ para $x = 0$

$|-2r(1 - \frac{1}{r}) + r| <= 1$ para $x = 1 - \frac{1}{r}$

$-1 <= -2r + 2 + r <= 1$

$-1 <= 2 - r <= 1$

$-3 <= - r <= -1 $

$1 <= r <= 3$

o que explica porque o método de interação simples não encontrou raízes
para $r=3.2$

\end{document}
