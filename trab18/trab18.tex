
%%% cabecalho %%%
\documentclass[brazilian,12pt,a4paper,final]{article}
\usepackage[brazil]{babel}% *babel* contem as regras de ifenização
%\usepackage{t1enc}% *t1enc* permite o reconhecimento dos acentos do teclado
\usepackage[utf8]{inputenc}% permite reconhecimento automático de acentuação.
\usepackage{graphicx} % para incluir figuras em formato eps 
%ou
%\usepackage[pdftex]{graphicx}% para produzir PDF diretamente
\usepackage{color} % fontes soloridas
%%% fim do cabecalho %%%

\pagestyle{empty}
%\title{Métodos Computacionais da Física A \\ Trabalho 15}
%\author{Aluno: Átila Leites Romero \\ Matrícula: 144679 \\ IF-UFRGS}

\begin{document}
%\maketitle

$S = (y_i-({\alpha}_0 +{\alpha}_1 x+ {\alpha}_2 x^2))^2$

$\frac{\delta S}{\delta {\alpha}_0} = 2(y_i-({\alpha}_0 +{\alpha}_1 x+ {\alpha}_2 x^2))(-1)$

$\frac{\delta S}{\delta {\alpha}_1} = 2(y_i-({\alpha}_0 +{\alpha}_1 x+ {\alpha}_2 x^2))(-x)$

$\frac{\delta S}{\delta {\alpha}_2} = 2(y_i-({\alpha}_0 +{\alpha} 1 x+ {\alpha}_2 x^2))(-x^2)$

----

$0 = 2(y_i-({\alpha}_0 +{\alpha}_1 x+ {\alpha}_2 x^2))(-1)$

$0 = 2(y_i-({\alpha}_0 +{\alpha}_1 x+ {\alpha}_2 x^2))(-x)$

$0 = 2(y_i-({\alpha}_0 +{\alpha} 1 x+ {\alpha}_2 x^2))(-x^2)$

----

$0 = (y_i-({\alpha}_0 +{\alpha}_1 x+ {\alpha}_2 x^2))(-1)$

$0 = (y_i-({\alpha}_0 +{\alpha}_1 x+ {\alpha}_2 x^2))(-x)$

$0 = (y_i-({\alpha}_0 +{\alpha} 1 x+ {\alpha}_2 x^2))(-x^2)$

----

$0 = ({\alpha}_0 +{\alpha}_1 x+ {\alpha}_2 x^2)(1)+(-1)(y_i)$

$0 = ({\alpha}_0 +{\alpha}_1 x+ {\alpha}_2 x^2)(x)+(-x)(y_i)$

$0 = ({\alpha}_0 +{\alpha} 1 x+ {\alpha}_2 x^2)(x^2)+(-x^2)(y_i)$

----

$({\alpha}_0 +{\alpha}_1 x+ {\alpha}_2 x^2)(1)=(1)(y_i)$

$({\alpha}_0 +{\alpha}_1 x+ {\alpha}_2 x^2)(x)=(x)(y_i)$

$({\alpha}_0 +{\alpha} 1 x+ {\alpha}_2 x^2)(x^2)=(x^2)(y_i)$

----

$
\left|\begin{array}{ccc}
1   & x   & x^2\\
x   & x^2 & x^3\\
x^2 & x^3 & x^4\\
\end{array} \right|
\left|\begin{array}{c}
{\alpha}_0\\
{\alpha}_1\\
{\alpha}_2\\
\end{array} \right|
= 
\left|\begin{array}{c}
1 y_i\\
x y_i\\
x^2 y_i\\
\end{array} \right|
$
\end{document}
